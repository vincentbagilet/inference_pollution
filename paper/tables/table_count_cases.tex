\begin{table}[!ht]
    \caption{Evolution of Power and Exaggeration with the Average Number of Daily Cases of Health Outcomes.}
    \label{tab:n_cases}
    \centering
            \begin{threeparttable}
                \begin{tabular}{lccc}
                    \toprule
                                         & Non-Accidental & Respiratory & COPD \\ \midrule
                                          %\rowcolor{Gray}
Number of Cases & 23                       & 2                     & 0.3                          \\
Statistical Power (\%)  & 90                       & 16                    & 7.5                          \\
%\rowcolor{Gray}
Exaggeration Ratio      & 1                        & 2.4                   & 5.9                          \\ \bottomrule
                \end{tabular}
                \begin{tablenotes}
                    \footnotesize
                    \item \textit{Notes}: This table displays the average number of cases, the power and the exaggeration ratio for three health outcomes: non-accidental deaths, respiratory deaths, and chronic pulmonary deaths for individuals aged between 65 and 75. These figures are obtained for the instrumental variable design with a sample size of 100,000 and 50\% of observations subject to an exogenous shock. The instrument variable increases the air pollutant concentration by 0.5 standard deviation. A one standard deviation increase in the instrumented air pollutant leads to 1\% relative increase in the health outcome considered.
                \end{tablenotes}
            \end{threeparttable}
\end{table}