\begin{table}[!ht]
	\caption{Policy Implications of Documented Exaggeration.}
	\label{tab:policy_implications}
	\centering
		\begin{threeparttable}
                		\begin{tabular}{lccccc}
                    		\toprule
                    			 & Exaggeration & Expected & Actual & Difference \\ 
			 		 & ratio & Net Benefits & Net Benefits & (B\$/year) \\ 
			 		 &  &  (B\$/year) & (B\$/year) & \\ 
				\midrule
				Epidemiology median & 1.3 & 45.4 & 34.9 & 10.5 \\
				Epidemiology 25th pctile & 1.9 & 45.4 &  23.9 & 21.5 \\
				Causal inference median & 1.7 & 45.4 & 26.7 & 18.7  \\
				Causal inference 25th pctile & 2.0 & 45.4 & 22.7 & 22.7 \\
				IV vs.\ OLS median & 4.5 & 45.4 & 10.1 & 35.3 \\
                    		\bottomrule
                \end{tabular}
                \begin{tablenotes}
                    \footnotesize
                    \item \textit{Notes}: Actual net benefits represent the net benefits under the hypothesis of exaggerated health effects: due to the linearity of the relationship between health effects and monetized benefits, they are simply computed as the expected net benefits of \$46 billion (the high estimate for the 9 $\mu\text{g/m}^3$ standard) divided by the exaggeration ratio found in our analysis of the literature. The last column represents the difference between the expected and actual net benefits.
                \end{tablenotes}
            \end{threeparttable}
\end{table}