This paper identifies tangible design parameters that might lead to inaccurate estimates of relatively small effects. Low statistical power not only makes such effects difficult to detect but resulting published estimates also exaggerate true effect sizes. Through the literature on the short-term health effects of air pollution, we explore the prevalence of this issue, its implications for public policy and identify its drivers using real data simulations replicating prevailing identification strategies used in economics. We find that while some studies seem robust, others lead to substantial exaggeration. Simulations reveal five key drivers of exaggeration: sample size, effect magnitude, proportion of exogenous shocks, instrument strength, and outcome distribution. While the analysis builds on a specific literature, it draws out insights that expand beyond this setting. Finally, we discuss approaches to evaluate and avoid exaggeration when conducting a non-experimental study.


%This paper identifies tangible design parameters that might lead to inaccurate estimates of relatively small effects. Low statistical power not only makes relatively small effects difficult to detect but resulting published estimates also exaggerate true effect sizes. Through the case of the literature on the short-term health effects of air pollution, we explore the prevalence of this issue, its implications for public policy and identify its drivers using real data simulations replicating most prevailing identification strategies used in economics. While the analysis builds on a specific literature, it draws out insights that expand beyond this setting. Finally, we discuss approaches to evaluate and avoid exaggeration when conducting a non-experimental study.